\documentclass[]{article}
\usepackage{amsmath,amsthm,amssymb,amsfonts}
\usepackage[shortlabels]{enumitem}
\usepackage{xcolor}

\setlength{\parindent}{0pt}

\usepackage{geometry}
\geometry{
	left=2cm,
	right=2cm,
}


% Package used for creating hyperlinks
%\usepackage{hyperref}
%\hypersetup{
	%	colorlinks=true,
	%	linkcolor=blue,
	%	citecolor=red,
	%	urlcolor=blue,
	%	pdfnewwindow=true
	%}

% Package for graphs, diagrams, and charts
\usepackage{tikz}
\usepackage{tikz-cd}
\graphicspath{ {./media/images/} }

\usepackage{graphicx}
\usepackage{animate}

\renewcommand{\Pr}[1]{\text{Pr}\left(#1\right)} % For probability to look like Pr(...)

\let\oldemptyset\emptyset % Rebind the crappy looking emptyset to another variable
\let\emptyset\varnothing % and redfine emptyset as the good looking one

\let\oldphi\phi % Rebind the crappy looking phi to another variable
\let\phi\varphi % and redefine phi as the good looking one


\newcommand\<{\ensuremath{\left\langle}}
\renewcommand\>{\ensuremath{\right\rangle}}
% Common math environments
\newcommand{\kk}{\ensuremath{\Bbbk}} 
\newcommand{\CC}{\ensuremath{\mathbb{C}}} 
\newcommand{\NN}{\ensuremath{\mathbb{N}}}
\newcommand{\QQ}{\ensuremath{\mathbb{Q}}} 
\newcommand{\RR}{\ensuremath{\mathbb{R}}}
\newcommand{\FF}{\ensuremath{\mathbb{F}}}
\newcommand{\ZZ}{\ensuremath{\mathbb{Z}}} 
\newcommand{\cO}{\ensuremath{\mathcal{O}}} 
\newcommand{\Aut}{\ensuremath{\mathrm{Aut}}} 
\newcommand{\Gal}{\ensuremath{\mathrm{Gal}}} 


% Solution environment. Basically just a standard proof environment but titled as "solution"
\newenvironment{solution}
{
	\begin{proof}[Solution] \text{ }
		\\
	}
	{
	\end{proof}
}

\newenvironment{graybox}{%
	\begin{lrbox}{\grayboxcontent}%
		\begin{minipage}{\textwidth}%
		}{%
		\end{minipage}%
	\end{lrbox}%
	\colorbox{gray!20}{\usebox{\grayboxcontent}}%
}

\newsavebox{\grayboxcontent} % Create a savebox to store the content

%Light gray box is for subproblems
\newenvironment{lightgraybox}{%
	\begin{lrbox}{\lightgrayboxcontent}%
		\begin{minipage}{\dimexpr\linewidth-2\fboxsep-2\fboxrule}%
		}{%
		\end{minipage}%
	\end{lrbox}%
	\colorbox{gray!10}{\usebox{\lightgrayboxcontent}}%
}

\newsavebox{\lightgrayboxcontent} % Create a savebox to store the content

\theoremstyle{definition}
\newtheorem{definition}{Definition}[section]

\theoremstyle{definition}
\newtheorem{example}{Example}[section]


\usepackage{verbatim}   % for the comment environment
\usepackage{ifthen}
\usepackage{lipsum}

%\newcommand{\showfigure}{show figs}

\newenvironment{displayfigure}[1][]
{%
	\ifthenelse{\isundefined{\showfigure}}%
	{\expandafter\comment}%
	{\begin{figure}[#1]}%
	}%
	{%
		\ifthenelse{\isundefined{\showfigure}}%
		{\expandafter\endcomment}%
		{\end{figure}}%
}%

%opening
\title{A Visual Introduction to Curvature}
\author{Evan Curry Wilbur}




%
% THE FOLLOWING ANIMATIONS HAVE BEEN CREATED, BUT NOT ADDED TO DOCUMENT
%     - SphereTangent   


\begin{document}
	\maketitle
	\begin{abstract}
		\indent
		This paper seeks to cover what one would learn in a years long graduate course in Riemannian geometry in a way that is both visually stimulating and enlightening. It is directed primarily to early undergraduate students of math who have familiarity with topics from linear algebra. Much reliance will be given to the reader's understanding of inner product spaces, determinants, linear transformations, as well as a familiarity with the geometric understanding of all these ideas. However, a precocious high schooler could follow along after reading chapters 1,2,3, and 6 of Shelden Axler's Linear Algebra Done Right. As well, practice in computing derivatives and integrals and the geometric meaning therein would be helpful. However, a course in vector calculus is not required.
		\\
		\\
		\indent
		By no means should this be seen as a full and rigorous exploration into the vast topic of Riemannian and differential geometry; many tomes have be written doing such a task. Instead, it will leverage visual intuition of the space we occupy to make broad and powerful statements about surfaces more generally. Of course, visual intuition is limited by the dimension we occupy, so many generalizations will be stated, without proof, with references to where one could find proofs.
	\end{abstract}
	
	\section{Making a Manifold}
	Any proper discussion of curvature must first answer the question \textit{what is being curved?} After all, asking how curvy the integers are is meaningless, while asking the same for a sphere is seemingly not. The difference lies in the notion of space and geometry for the latter that the former lacks. A language capturing this intuition is necessary if any progress is to be made in the theory of curvature. Thankfully, mathematicians have developed such a language for discussing all manner of surfaces. A complete discussion of the theory and language would require an exploration into topology in a way more rigorous than necessary in developing an intuition for the subject at hand, so instead, many of the gaps in rigor should be filled in by the reader's intuition.\\
	\\
	We will start the discussion by giving the definition for charts, which intuitively we will think of as a GPS system, or coordinates, allowing us to ``walk along'' a manifold in a way similar to how a spaceship would navigate through the solar system or how an ant would walk along a table. It will be clear, by how we define charts, that they are not suitable by themselves to encapsulate even all the surfaces we encounter in our everyday lives, much less those of higher dimensions. \\
	\\
	To remedy this, we will ``stitch together'' different charts to create a manifold, the central object of discussion for the remainder of the paper. Once some examples of manifolds have been given, we will begin discussing idea of smoothness and differentiability on a manifolds.
	\subsection*{Charts}
	At its most distilled, a chart is a map of some space around a point. Anyone who has picked up a map, used a GPS, or fantasized being a navigator in Star Trek is already familiar with the idea of charts. The chart is not the object of study in and of itself, much in the same way a map is not the area it seeks to represent. Rather, with a chart in hand we may point to a position on it, and we shall get a corresponding point on the surface we are studying. The animation below gives an intuition of how a chart can map out part of the surface $z = \sin(x + y)$.
	\begin{displayfigure}[h]
		\centering
		\animategraphics[loop,autoresume,autopause,autoplay, width=\textwidth]{20}{media/animations/chart1/chart/frame-}{16}{110}
	\end{displayfigure}
	\\
	From the animation, we can start to make some observations and develop an intuition for what charts do. First, they make the curvy space appear flat. The chart exists in $\RR^2$, and so is flat, yet despite this, it represents a rather curvaceous function! For those who have held a map or used GPS, this should seem like a natural requirement to have for charts. After all, if a map were required to conform to the natural topography of the region it's meant to represent, it would be rather cumbersome to use. Another thing one might notice is how the path traced out on the chart corresponds rather well to the other path on the surface. Again, this should feel like a natural property that charts should have. If a path followed on a map doesn't match a similar path traced on a surface, then its a bad map. The same holds true for charts. With these observations in mind, we give the formal definition for a chart.
	\begin{definition}
		A chart for a topological space $M$ is a homeomorphism from an open set $U \subseteq \RR^n$ to an open set $V \subseteq M$.
	\end{definition}
	\noindent
	For those without an introduction into topology, this definition is likely meaningless. There are three main pieces of vocabulary to unpack: \textit{topological spaces}, \textit{homeomorphism}, and \textit{open sets}, which will be done shortly. Before that, however, a quick disclaimer. The definitions for these terms will not be the same as one is likely to find in a topology course. They are deliberately tailored and narrowed to fit the scope of this paper. While the definitions will surly differ from a proper topology course, within the scope of the paper they will be equivalent. 
	\begin{definition}
		An open ball of radius $\varepsilon \in \RR$ centered at $p \in \RR^n$ is the set $B_p(\varepsilon) = \{x : x \in \RR^n, |x - p| < \varepsilon\}$. When $p$ is omitted from the expression, it is assumed to be centered at the origin.
	\end{definition}
	\begin{definition}
		A set $U \subseteq \RR^n$ is open if for every $p \in U$ there exists an $\varepsilon \in \RR$ such that $B_p(\varepsilon) \subseteq U$.
	\end{definition}
	\noindent
	An open set is one where around every point there is a open ball which remains in the set. Intuitively, something living inside an open set would always be able to stretch out their arms at least a little, no matter where they are.
	\begin{example}
		An open ball in $\RR^2$ of radius 1 centered at the origin is open.
		\\
		\begin{displayfigure}[h]
			\centering
			\animategraphics[loop,autoresume,autopause,autoplay, width=0.5\textwidth]{20}{media/animations/open-ball/open-ball/frame-}{1}{90}
		\end{displayfigure}
		\\
		The dashed line indicates that we are concerned with all the points up to the boundary. Notice as we move the dot around the ball, we may always shrink the red circle so that it lies entirely inside the ball.
	\end{example}
	\begin{example}
		The closed ball of radius 1 $\bar{B} = \{x : x \in \RR^2, |x| \leq 1\}$ is not open.
	\end{example}
	\begin{displayfigure}[h]
		\centering
		\animategraphics[loop,autoresume,autopause,autoplay, width=0.5\textwidth]{15}{media/animations/closed-ball/closed-ball/frame-}{1}{30}
	\end{displayfigure}
	Unlike in the previous example, this ball contains its boundary. No matter how small of a circle you draw around a point on the boundary, there will always be a part that remains outside the original circle. Therefore, the closed ball is not open.
	\\
	\\
	With an understanding what an open set is, we are now ready to give the definition for a topological space
	\begin{definition}
		A topological space is a set $M$ along with a collection $\tau$ of open sets of $M$ such that
		\begin{enumerate}[1.]
			\item Both $M$ and the empty set are in $\tau$. Succinctly, $M \in \tau, \emptyset \in \tau$.
			\item For any index set $U_i \in \tau, \hspace*{5px}\cup U_i \in \tau$. That is, an arbitrary union of open is open.
			\item For any finite collection $U_k \in \tau, \hspace*{5px}\cap^n_{k=0} U_k \in \tau$. That is, finite intersections of open sets are open.
		\end{enumerate}
	\end{definition}
	This definition is abstract by design. It is meant to encapsulate many possible, even unimaginable, topological spaces. However, for our purposes we need only translate this to our language with open balls and the ideas will be more clear
	\begin{definition}
		A topological space is a set $M \subseteq \RR^n$ with a collection $\tau$ of open balls of $M$ such that
		\begin{enumerate}[1.]
			\item The open balls $B(0) = \emptyset$ and $B(\infty) = \RR^n$ are in $\tau$.
			\item Any union of open balls will be open.
			\item Any finite intersection of open balls will be open.
		\end{enumerate}
	\end{definition}
	
	Finally, we get to the last of the vocabulary to unpack, \textit{homeomorphism}. When we think of two objects being homeomorphic, then we may continuously transform one into the other without ripping, tearing, or gluing our shape, and then back again.\\
	\begin{displayfigure}[h]
		\centering
		\begin{minipage}{0.5\textwidth}
			\animategraphics[loop,autoresume,autopause,autoplay, width=0.65\textwidth]{20}{media/animations/coffee-mug-donut/mug-torus/frame-}{1}{116}
		\end{minipage}%
		\begin{minipage}{0.5\textwidth}
			\animategraphics[loop,autoresume,autopause,autoplay, width=1.5\textwidth]{13}{media/animations/homeo/homeo/frame-}{1}{75}
		\end{minipage}
	\end{displayfigure}\\
	
	Due to the animation on the left, mathematicians often joke that the topologist cannot distinguish between a coffee cup and a donut since the two are homeomorphic. While said in jest, it does underpin a critical point in topology. When two objects are homeomorphic to each other \textit{they are functionally the same object}.\\
	\\
	Another important idea illustrated in the animations is that, while we cannot rip, tear, or otherwise damage the object during a homeomorphism, we are freely allowed to stretch, warp, and even self-intersect as much as we would like. With that, here is the definition
	\begin{definition}
		A function $f : M \to N$ between topological spaces $M$ and $N$ is a homeomorphism if
		\begin{enumerate}[1.]
			\item $f$ is bijective.
			\item $f$ is continuous.
			\item $f^{-1}$ is continuous.
		\end{enumerate}
	\end{definition}
	These three conditions encapsulate exactly the conditions mentioned previously, just in the language of topology. The requirement that $f$ and $f^{-1}$ be continuous forbids us from ripping, tearing, and otherwise damaging our space, with $f^{-1}$ also allowing us to ``undo'' the transformation. The requirement that $f$ is bijective means that we cannot ``squish'' open sets of our surface down to a point. It is possible to continuously transform a circle down to a point, there is no way to do this without collapsing the open sets of the circle. In the animation below, the parentheses indicate one of the open sets of the circle. It's clear that as the circle transforms continuously to a point, the open set collapses to a point as well. This violates (1) of the definition of homeomorphism, thus it cannot be a homeomorphism, despite being a continuous transformation.
	\begin{displayfigure}[h]
		\centering
		\animategraphics[loop,autoresume,autopause,autoplay, width=0.5\textwidth]{10}{media/animations/circle-dot/circle-dot/frame-}{1}{30}
	\end{displayfigure}\\
	
	With the language out of the way, we now are in a position to understand charts in a way much more clear than the rigorous definition.
	\begin{definition}
		A chart is an open ball $U \subseteq \RR^n$ representing some surface $M$ in such a way that $U$ can be continuously transformed into $M$ and back again.
	\end{definition} 
	In essence, a chart is a flat map of a (potentially) curvy space.
	\newpage
	\begin{example}
		The following animations illustrate a variety of charts and the surfaces they correspond to. Observe how they are all able to be continuously deformed into a flat surface and then back again.
		\begin{displayfigure}[h]
			\begin{minipage}{0.5\textwidth}
				\centering
				\animategraphics[loop,autoresume,autopause,autoplay, width=0.85\textwidth]{60}{media/animations/helicoid/helicoid/frame-}{17}{300}
				\caption{Helicoid}
			\end{minipage}
			\begin{minipage}{0.5\textwidth}
				\animategraphics[loop,autoresume,autopause,autoplay, width=0.85\textwidth]{12}{media/animations/monkey-saddle/monkey-saddle/frame-}{1}{30}
				\caption{Monkey Saddle}
			\end{minipage}\\
			\begin{minipage}{0.5\textwidth}
				\animategraphics[loop,autoresume,autopause,autoplay, width=0.85\textwidth]{12}{media/animations/paraboloid/paraboloid/frame-}{1}{30}
				\caption{Paraboloid}
			\end{minipage}
			\begin{minipage}{0.5\textwidth}
				\animategraphics[loop,autoresume,autopause,autoplay, width=0.85\textwidth]{12}{media/animations/hemisphere/hemisphere/frame-}{1}{30}
				\caption{Hemisphere}
			\end{minipage}
			\begin{minipage}{0.5\textwidth}
				\animategraphics[loop,autoresume,autopause,autoplay, width=0.85\textwidth]{12}{media/animations/hyperbolic-paraboloid/hyperbolic-paraboloid/frame-}{1}{30}
				\caption{Hyperbolic Paraboloid}
			\end{minipage}
			\begin{minipage}{0.5\textwidth}
				\animategraphics[loop,autoresume,autopause,autoplay, width=0.85\textwidth]{12}{media/animations/hyperboloid/hyperboloid/frame-}{1}{30}
				\caption{Hyperboloid}
			\end{minipage}
		\end{displayfigure}
	\end{example}
	\newpage
	\subsection*{Manifolds}
	In the previous section, we saw that we were able to construct many surfaces using just a single chart. It might be tempting to think that a theory of charts is all we need to study all conceivable surfaces. This, however, is not the case. The issue lies in the requirement that our charts be homeomorphic to the space they represent. This is a strong requirement that prevents us from studying many surfaces which may not be homeomorphic to flat space with charts alone.
	\begin{example}
		The sphere is not homeomorphic to flat Euclidean space, thus it cannot be represented by a single chart.
		\begin{displayfigure}[h]
			\animategraphics[loop,autoresume,autopause,autoplay, width=\textwidth]{30}{media/animations/sphere-not-flat/sphere-not-flat/frame-}{1}{75}
		\end{displayfigure}
	\end{example}
	Illustrated in the animation is the unfortunate fact that we cannot transform the sphere into a flat surface without overlapping two different regions on the sphere onto the same region of a chart, in violation of rule (1) of the definition of homeomorphism. If the sphere were missing one of the poles, then it would in fact be possible to cover it in one chart. However, this requires that we puncture the sphere, a violation of rule (2). This is by no means meant to be a proof of the fact that we cannot cover the sphere in a single chart, rather it is only meant to illustrate what goes wrong when we try to. A curious reader would find that with only a brief exploration into topology, and in particular Brouwer's fixed point theorem, the proof that it is impossible to cover the sphere in a single chart is rather quick and enlightening.\\
	\\
	Since the theory of charts is not enough to study surfaces as typical and canonical as the sphere, we may wonder if is even worth talking about them at all. Is all hopelessly lost? Did the author really just spend the past $\thepage$ pages talking about something that has no bearing on the study of surfaces? Of course not. As is so common in mathematics, when our theory runs into a wall, we extend the theory so we may be able to walk up walls. \\
	\\
	The key idea is not to limit ourselves by using only a \textit{single} chart. Instead, we will use multiple charts to cover our surfaces. Each chart will represent a local patch of the surface. This local patch, by virtue of being covered by a chart, will look like an open set of Euclidean space. Because of this, we say that our surfaces are \textit{locally Euclidean}. A collection of charts which cover a surface, when taken in aggregate, will be powerful enough to describe a vast amount of spaces. This collection is called an \textit{atlas}, and the space which they cover is called a \textit{manifold}. With that, we present two definitions.
	\begin{definition}
		An atlas is a collection of charts of the same dimension.
	\end{definition}
	\begin{definition}
		An $n$-dimensional manifold $M$ is a topological space such that for each point $p \in M$, there is a chart $U \subseteq \RR^n$ and a homeomorphism $\phi : U \to M$ such that $p \in \phi(U)$.
	\end{definition}
	With this definition now firmly established, we will cease to use the terms "surfaces" and instead use the more general term, manifold. The key idea here is that by considering collections of charts, we can study a vast collection of topological spaces.
	\begin{example}
		By covering the sphere in two charts, we are able to turn it into a manifold.\\
		\begin{displayfigure}[h]
			\centering
			\animategraphics[loop,autoresume,autopause,autoplay,palindrome, width=\textwidth]{30}{media/animations/sphere-manifold/sphere-manifold/frame-}{1}{75}
		\end{displayfigure}
	\end{example}
	The animation shows that the charts of a manifold may overlap. On the region where charts overlap, we will have points which can be described by more than one chart. These regions of overlap will have a special consideration when we discuss smooth manifolds in the next section. For now, we will only consider how we can move between two charts which describe the same region of our manifold. 
	\begin{definition}
		Given two charts $(U, \phi), (V, \psi)$, of a manifold $M$, if $\phi(U) \cap \psi(V) \neq \emptyset$, then the transition map is the homeomorphism $\phi^{-1} \circ \psi : \RR^n \to \RR^n$.
	\end{definition}
	\begin{example}
		Shown is the manifold $M$ with two charts. The non-overlapping regions of $V$ and $U$ are colored red and blue respectively. Their shared region is colored green. We see how the transition map $\phi^{-1} \circ \psi$ maps \textit{between the charts}.
		\begin{displayfigure}[h]
			\centering
			\includegraphics[scale=0.33]{transition-map/transition-map.png}
		\end{displayfigure}
	\end{example}
	
	
	\subsection*{Smooth Manifolds}
	So far, our discussion has been centered around building up an intuition and language for manifolds. We have actually answered the question posed at the very beginning of this paper, \textit{what is being curve}. The answer: manifolds. As is so common in mathematics, when one question is answered, others naturally arise. While we now know \textit{what} is being curved, we do not yet know \textit{how} it is being curved. Unfortunately, we are not in a position to answer this question just yet. The issue is that our definition for a manifold is not powerful enough to discuss topics as complex as curvature.\\
	\\
	If you recall, the way we built up to the definition of manifolds was my stitching together regions of flat space. We then were able to discuss and study manifold by referencing their charts. However, the charts are just open sets of Euclidean space. Not only are they flat, they are essentially as flat as mathematically possible. So how can we possibly discuss curvature with only our collection of charts? The answer is that we can't. The definition is simply not capable of expressing these ideas.\\
	\\
	Not all is lost, however. Much like we were able to extend our definition of charts to manifolds and expand the types of topological spaces we could talk about, we can build upon our current definition of manifolds in a way powerful enough to discuss curvature.\\
	\\
	Before we do that, however, let us see what goes wrong when using our current definition of manifolds. To do so, let us first introduce the concept of a path. We understand intuitively that a path is just some trajectory through space that doesn't have gaps or holes in it. The following definition expresses mathematically this idea
	\begin{definition}
		Let $M$ be a manifold. A path is a continuous function $[0, 1] \to M$.
	\end{definition}
	So a path is a function which maps a line segment continuously over a manifold.\\
	\\
	Recall that our overall goal is to develop a theory so we may discuss curvature. Curvature, we understand, is a bending or change of direction of an object (or in our case, manifolds). However, it's not just any change of direction or bending. For example, a zig-zag path changes direction frequently, however, one wouldn't consider it ``curving'' through space. For something to be curved, it requires a smooth change of direction. \\
	\begin{example}
		Both paths follow a very similar trajectory through space. However, the path on the left consists of sharp and jagged turns while the one on the right takes smoother turns. The one on the left does not appear to curve while the one on the right does.
		\begin{displayfigure}[h]
			\centering
			\includegraphics[scale=0.4]{curve-and-zig-zag/curve-and-zig-zag.png}
		\end{displayfigure}
	\end{example}
	Let's imagine a person walking along the zig-zag curve on the left. At first, they will walk in a straight line until they reach the first corner. When they arrive at the corner, they must pause, change their direction, and then proceed walking. Compare this to that same person walking along the path on the right. Unlike before, they do not need to stop in order to change direction. They instead are able to walk along the path while slowly adjusting their direction as they walk.
	
	
	
	
	\newpage
	\begin{example}
		One key observation that we can make is how the direction the dot on the zig-zag path is moving changes abruptly when it reaches a corner. Compare this to the dot moving along the smooth curve. The direction it is moving does not change abruptly. If we wish to talk about curvature, we require that these tangent directions do not change abruptly.
		\begin{displayfigure}[h]
			\centering
			\animategraphics[loop,autoresume,autopause,autoplay,width=\textwidth]{30}{media/animations/direction-path/direction-path/frame-}{1}{75}
		\end{displayfigure}
	\end{example}
	Let's explore this idea a bit more as it is essential to understanding smooth manifolds and curvature more broadly. In a first year calculus course, we learn of the derivative and how it is used to calculate the slope of the tangent line to functions. The tangent line indicates the instantaneous direction of the function at any given point. So, the directions indicated by the arrows in Example 1.8 are nothing more than the derivative of the paths, with additional information indicating the direction of travel. In fact, the derivatives of the equations for the paths were exactly what were used to compute the direction of the arrows in the animations!
	\begin{definition}
		Let $M$ be a manifold. The derivative of a path $f : [0, 1] \to M$ at time $t_0 \in [0,1]$ is a vector. The direction of the vector indicates the direction of travel along the path as a particle moves through $t_0$ from $[0, 1]$. The magnitude of the vector is the speed of the particle at time $t_0$. 
	\end{definition}
	\begin{definition}
		A smooth path is a path whose derivative is continuous. That is, the direction and speed of the derivative are continuous.
	\end{definition}
	The problem with the zig-zag path can now be stated more mathematically. The corners where we make abrupt changes in direction are discontinuities of the derivative of the path. Thus, we shall require that all paths have continuous derivatives. In other words, the vector along a path has no sudden or abrupt changes in speed or direction.\\
	\\
	We are now in a position to define what a smooth manifold is by utilizing the notion of a smooth path. We understand that a manifold can be realized as a collection of charts. If we have a smooth path on a chart, then for a manifold to be smooth it will be necessary that the corresponding path on the manifold be smooth as well.
	\begin{example}
		Despite the fact that the path on the chart is smooth, because the manifold has sharp corner, the corresponding path is not smooth. Thus, the manifold is not smooth.
		\begin{displayfigure}[h]
			\centering
			\includegraphics[width=0.3\textwidth]{nonsmooth-manifold/nonsmooth-manifold.png}
		\end{displayfigure}
	\end{example}
	\begin{example}
		The sphere is a smooth manifold. Every smooth path drawn on a chart of the sphere is itself smooth.
		\begin{displayfigure}[h]
			\centering
			\includegraphics[width=0.2\textwidth]{smooth-manifold/smooth-manifold.png}
		\end{displayfigure}
	\end{example}
	With respect to our discussion of smooth paths, it might be tempting to define a smooth manifold as one where every smooth path on a chart produces a smooth path on the manifold. Intuitively, this idea is correct. However, it turns out that when defining these concepts more rigorously, such a definition requires us to be able to compute the derivatives of these paths on the manifold. This isn't a problem with the manifolds we've seen so far because we've been visualizing all our manifolds as being embedded in a surrounding space. Mathematicians, however, prefer to divorce the idea of a manifold and the space which it occupies. They wish to treat the manifold \textit{as a space unto itself}. If we were to accept this definition for smooth manifolds, it would require that we reinvent calculus on every manifold to define what it even means to take a derivative on a manifold. To avoid undertaking such an insurmountable task just to define what it means for a surface to be smooth, mathematicians instead consider an alternative approach involving the differentiability of transition maps between charts of a manifold. Since we understand what it means for a function to be differentiable in $\RR^n$, and transition maps are simply functions $\RR^n \to \RR^n$, we can define a smooth manifold as one where the transition maps are differentiable. Realize, however, that defining smoothness via the transition maps or by requiring that smooth paths on charts be smooth on the manifold gives exactly the same realization of smoothness. 
	
	\begin{definition}
		An $n$-dimensional manifold $M$ is smooth if, whenever two charts $(U, \phi)$ and $(V, \psi)$ of $M$ overlap, the transition map $\phi^{-1} \circ \psi : \RR^n \to \RR^n$ is differentiable. 
	\end{definition}
	
	
	
	
	\newpage
	\section{The Measure of a Manifold}
	So far, it seems that every time we introduce a concept to help us understand curvature, the rug gets pulled and it turns out that what we've been studying wasn't \textit{quite} enough to speak on the nature of curvature. First, the notions of charts were discussed and studied. However, it was pointed out that charts alone weren't enough to discuss many surfaces that we would otherwise wish to. To remedy this, we bundled and stitched charts together to create manifolds. We then saw how the definition of a manifold was actually too broad to have a coherent discussion on curvature. It allowed for manifolds with sharp curves or edges, which we pointed out were not the curvature we are interested in studying. So we refined the definition of the manifold to one more suitable to a discussion of curvature: smooth manifolds. Exasperation would not be an unreasonable emotion for a reader at this point. Thankfully, it turns out that smooth manifolds are sufficient for our purpose of understanding curvature.\\
	\\
	This section will be dedicated to developing tools to let us better understand smooth manifolds. We will start with defining the tangent space to a manifold. This idea is central to a proper understanding of smooth manifolds as it allows us to extend ideas from single variable calculus to any smooth manifolds. This will let us utilize the tools and concepts that calculus provides us to better understand curvature.\\
	\\
	We will explore more deeply the connection between the tangent space of a manifold and the underlying charts which define it through the concept of the pushforward, or differential, map. With this, we can compare directions on the charts for a manifold to the manifold itself by ``pushing them forward''. \\
	\\
	Once we have defined the tangent space to a manifold and see how it compares to the tangent space of the underlying charts, we will realize that it is nothing more than a vector space attached to each point on a smooth manifold. Because of this, we can utilize concepts from linear algebra, like inner product spaces, to better understand the structure of manifolds.
	
	\subsection*{The Tangent Space}
	In introductory calculus, we learn about derivatives, tangent lines, and their applications and interpretations. The derivative, we saw, gave us a function's instantaneous rate of change. Tangent lines were special lines which lie tangent to a graph at a given moment and serve as the best linear approximation for a function at that point. When we took the derivative \textit{of the derivative} of a function, we were able to see how quickly the function diverges from the tangent line. This idea will be essential later when we finally and properly study curvature. Before we do that, however, we will need to extend these ideas from introductory calculus more generally to manifolds. The tangent space of a manifold is the first step in doing this.\\
	\\
	Much like how the tangent line to a graph gave us a linear approximation of the graph at a given point, so too does the tangent space of a manifold give us a linear approximation to the manifold at a given point. In fact, we will see that the tangent space is simply a generalization of concept of a tangent line we've already seen.\\
	\\
	We can think of the graph of a real-valued function as a one dimensional manifold. If we were to attach at each point of the graph the tangent line at that point, we would begin to have something that looks like the tangent space of the graph.
	\newpage
	\begin{example}
		Attaching a tangent line to every point of a graph resembles the tangent space to the graph
		\begin{displayfigure}[h]
			\centering
			\animategraphics[loop,autoresume,autopause,autoplay,width=0.5\textwidth]{30}{media/animations/tangent-graph/tangent-graph/frame-}{1}{180}
		\end{displayfigure}
	\end{example}
	Similarly to the one dimensional example, we can extend this idea to that of two dimensions. However, instead of tangent lines we instead consider tangent planes.
	\begin{example}
		\text{ }
		\begin{displayfigure}[h]
			\centering
			\animategraphics[loop,autoresume,autopause,autoplay,width=0.50\textwidth]{30}{media/animations/sphere-tangent/sphere-tangent/frame-}{1}{150}
		\end{displayfigure}
	\end{example}
	Here we see the tangent plane to a sphere taken at various points. The plane always lies tangent to the sphere, and so serves as a linear approximation to the sphere at any given point. This idea generalizes to any smooth manifold. In fact, the primary motivation for requiring our manifolds to be smooth was so we can have a well defined tangent space at every point of the manifold. As the manifold increases in dimension, so too does the dimension of the tangent space at a point.\\
	\\
	It might be tempting to define the tangent space as simply the manifold along with all the tangent spaces attached at each point. This isn't too far off from the actual definition. However, it turns out to be more effective to define it as a linear subspace at each point. In other words, rather than attaching planes and lines to the manifold at every point, we instead attach a vector space at every point.\\
	\\
	It should be clear to anyone who has taken linear algebra why this would be more useful. Lines and planes are simple geometric objects. However, each of these geometric objects have implicitly a vector space which spans them, and using vector spaces allows us to tap into the powerful tools in linear algebra.\\
	\\
	This begs the question: how exactly do we define a vector space at every point of the manifold. Intuitively, we know what needs to be done. However, using only the charts we have which define our manifold, it's not immediately obvious how we should go about constructing this. There are actually many ways to do this rigorously. We will show one here which uses the tangent space of paths to build a tangent space to a point, but we will avoid the rigor in lieu of building the intuition first. For the reader who wants a more detailed, thorough, and rigorous construction of the tangent space, John Lee's Introduction to Smooth Manifolds is a fantastic resource.\\
	\\
	We've seen some hints how we are going to go about constructing the tangent space from examples 1.8 and 1.9 from the previous section. The key idea is to leverage what we know about the tangent spaces of paths on manifolds to construct the tangent space at a point.\\
	\\
	We already were introduced to paths, derivatives of paths, and smooth paths in definitions 1.11, 1.12, and 1.13. These work as definitions, but they don't exactly tell us how to go about \textit{computing} these derivatives. Here we will realize how.
	\begin{example}
		The graph of the function $f:[0,1]\to \RR^3$
		$$
			f = \begin{pmatrix}
				\sin(2t\pi)\\\cos(2t\pi)\\\cos(4t\pi)
			\end{pmatrix}
		$$
		\begin{displayfigure}[h]
			\begin{minipage}{0.35\textwidth}
				\animategraphics[loop,autoresume,autopause,autoplay, width=1.5\textwidth]{30}{media/animations/particle-path/particle-path/frame-}{1}{75}
			\end{minipage}	
			\begin{minipage}{0.6\textwidth}
				\animategraphics[loop,autoresume,autopause,autoplay, width=\textwidth]{30}{media/animations/particle-path/particle-path-tex/frame-}{1}{75}
			\end{minipage}
		\end{displayfigure}
	\end{example}
	We see here an example of a parametrization of a path. You can see how we can plug in a variable from $[0, 1]$ and we will output a vector which points to the position on the curve. The parametrization of the path reifies the abstract notion of a path into one where we can actually work with and compute. With the parametrization of a path in hand, we can compute the derivative simply by computing the derivative of each component of the path. 
	\\
	\begin{example}
		The graph of the function $f : [0,1] \to \RR^3$ from Example 2.3 and its derivative
		\begin{align*}
			f = \begin{pmatrix}
				\sin(2t\pi)\\\cos(2t\pi)\\\cos(4t\pi)
			\end{pmatrix} && f' = \begin{pmatrix}
				2\pi\cos(2t\pi) \\ -2\pi\sin(2t\pi)\\-4\pi\sin(4t\pi)
			\end{pmatrix}
		\end{align*}
		
		\begin{displayfigure}[h]
			\begin{minipage}{0.35\textwidth}
				\animategraphics[loop,autoresume,autopause,autoplay, width=1.5\textwidth]{30}{media/animations/particle-path-tangent/particle-path-tangent/frame-}{1}{75}
			\end{minipage}	
			\begin{minipage}{0.6\textwidth}
				\animategraphics[loop,autoresume,autopause,autoplay, width=\textwidth]{30}{media/animations/particle-path-tangent/particle-path-tangent-tex/frame-}{1}{75}
			\end{minipage}
		\end{displayfigure}
	\end{example}
	The parametrization as well as its derivative allows us to express the idea of a tangent space using just the equations of the path. Let $\alpha$ be a path and let $\alpha'$ be the derivative of the path. At some point $t_0 \in [0,1]$, $\alpha(t)$ describes a position along the path while $\alpha'(t)$ gives the tangent direction of the path. With this, we are able to construct a one dimensional vector space $T$ at $\alpha(t_0)$ by
	$$
		T = \alpha(t_0) + \lambda\alpha'(t_0) \hspace{1cm} \lambda \in \RR.
	$$
	
	
	\newpage
	\begin{example}
		The graph of the function $f$ from Example 2.3 along with the tangent space at $t = \frac{1}{8}$ as we vary the scalar $\lambda \in \RR$
		\begin{align*}
			f\left(\frac{1}{8}\right) = \begin{pmatrix}
				\sin\left(\frac{\pi}{4}\right)\\\cos\left(\frac{\pi}{4}\right)\\\cos\frac{\pi}{2}
			\end{pmatrix} && f' = \begin{pmatrix}
				\cos(2t\pi) \\ -\sin(t\pi)\\-2\sin(2t\pi)
			\end{pmatrix}
		\end{align*}
		\begin{displayfigure}[h]
			\begin{minipage}{0.35\textwidth}
				\animategraphics[loop,autoresume,autopause,autoplay, width=1.5\textwidth]{45}{media/animations/particle-path-tangent-space/particle-path-tangent-space/frame-}{1}{90}
			\end{minipage}	
			\begin{minipage}{0.6\textwidth}
				\animategraphics[loop,autoresume,autopause,autoplay, width=\textwidth]{45}{media/animations/particle-path-tangent-space/particle-path-tangent-space-tex/frame-}{1}{90}
			\end{minipage}
		\end{displayfigure}
	\end{example}
	You can see illustrated in the example how we can attach a vector space at a point on the path. Doing this for every point along the path gives us the tangent space of the path.\\
	\\
	By taking the derivative of a smooth path, we've seen how we can construct a vector space for every point of the path. When taken in aggregate, the vector spaces become of the tangent space to the path, when we view the path as a manifold. We have the power now to construct the tangent space for every one dimensional smooth manifolds. This is all well and good. However, many interesting manifolds exist in higher dimensions and we've only seen how to construct a tangent space to one dimensional manifolds.\\
	\\
	One of mathematics' greatest strengths is its ability to generalize concepts to those not imagined previously. We will tap into that strength right now and extend the concept of a tangent space to manifolds of higher dimensions. To do this, we will build up the tangent space to a point on the manifold by considering \textit{multiple paths} through that point. When taken in aggregate, the paths will create a vector space, or tangent space, to the manifold at a point.\\
	\\
	Suppose we have an $n$-dimensional manifold $M$. We wish to build the tangent space on $M$ at a point $p \in M$. As a matter of notation, we will refer to this tangent space as $T_pM$. Since $p$ is a point on the manifold, there exists a chart $(U, \phi)$ and a point $u \in U$ such that $\phi(u) = p$. Consider now a smooth path $f : [0,1] \to U$ such that $f(t_0) = u$ for some $t_0 \in (0,1)$. We've seen already that the derivative of $f$ gives us the tangent direction of the path at a point. So then $f'(t_0)$ is the direction of $f$ through the point $u$. By using the map $\phi$ we can carry this path over to $M$. That is, the composition $\phi \circ f \subseteq M$ is just a path in $M$. But we know exactly how to compute derivatives of paths. So by computing the derivative of $\phi \circ f$ at $t_0$ gives us a tangent direction to the point $p$ on the manifold! We are able to do this for any smooth path passing through the point $u \in U$. By considering all paths through the point $u$ and considering the tangent vectors we find after the composition $\phi \circ f$, we build $T_pM$.
	\begin{example}
		Show a picture of a chart and a manifold. Show multiple paths through a point on the chart corresponding to a point on the manifold and show how these construct the tangent space.
	\end{example}
	
	Thus, the tangent space to a point can be thought of as the set of all paths going through a point on the chart. Doing this for every point on the manifold gives us the total tangent space to the manifold, regardless of the dimension.
	
	\subsection*{The Pushforward Map}
	We now have this additional structure attached to every point of a smooth manifold which we call the tangent space. As we saw, the tangent space contains information about how the manifold changes at a given point. In other words, it provides the best linear approximation of a manifold at a point. We will now explore the connection between the tangent space of a manifold and the underlying charts used to define the manifold via the pushforward map. \\
	\\
	Recalling our analogy for a chart as a map or GPS coordinates for a manifold, we can think of a pushforward map as a way of translating directions on the chart to directions on the manifold. That is, if we position ourselves somewhere on a chart describing a region of a manifold, we may draw vectors indicating directions we may travel on the chart. These directions correspond to directions on the tangent space of the manifold. In a way, the pushforward lets us translate directions on our chart to directions on our manifold. Thus, we can consider the pushforward as a map between the tangent space of a chart to the tangent space of a manifold.
	
	\subsection*{The Riemannian Metric}
	With the tangent space and the pushforward map, we have a way of understanding directions on the manifold by considering directions on a chart. As already mentioned, the tangent space of a manifold at a point is simply a vector space. Thus it would make sense to tap into our knowledge 
	
	
	
	\newpage
	\section{Curvature}
	Our theory is developed enough to finally begin exploring the main topic of this paper: curvature. In much the same way we built up a theory of manifolds starting from the basic charts, building all the way up to smooth manifolds, we develop our theory of curvature by studying how simple manifolds curve. We will then generalize these ideas to more complicated manifolds to broaden our theory. 
	
	\subsection*{One Dimensional Curvature}
	If we have any hope in understanding curvature in a general sense, we should start by studying curvature on simple manifolds and generalize the idea to more complicated ones. Naturally, a question to have would be \textit{what is a simple manifold?} For our study, the simplest manifolds we can work with are one dimensional manifolds. Recall from the definition that a manifold is something that ``looks'' locally Euclidean. Most of the examples we've seen so far have been two dimensional manifolds (surfaces), primarily because they provide the best insight for developing the intuition behind our theory as well as the most engaging visualizations for them. However, our discussion of curvature will begin with one dimensional manifolds. Conceptually, we think of these one dimensional manifolds as a path in space.
	\begin{example}
		Examples of curves and one dimensional manifolds. At least one of these examples needs to be a straight line and the other needs to be something curvy
	\end{example}
	As the phrase ``one dimensional manifolds'' can be quite a mouthful, we often refer to these objects as a ``path'' or ``curve''. 
	\begin{definition}
		A curve or path is a one dimensional manifold.
	\end{definition}
	In example 3.1, we saw some curves which appeared curvy, while others we decidedly straight. This notion of curvature seems intuitively obvious. So much so that even a small child will have no trouble categorizing which of the curves is curvy and which are not. However, just as we had an intuitive concept of what a surface was, but developed the theory of manifolds so we had a language and formalism to discuss them, we approach this concept with a little more rigor than simply our intuition.\\
	\\
	Imagine you are driving in a car along a straight highway at a constant speed. Our experience with the physical world tells us that you will not feel any force acting on you while driving. Eventually, you approach a curve. Without slowing down you take the turn. Suddenly, you feel a force towards the center of your turn. Your speed didn't change, but nonetheless a force was felt. After the turn, you continue driving straight ahead and resume not feeling any force on you. Thinking to yourself, you realize that as long as you maintain your speed, any force felt by you or the passengers would indicate a turn, or change of direction. Interestingly, even someone completely blindfolded, or not looking outside the car at all, could determine when they are turning and when they are not, simply by feeling the forces acting on them while inside the car. This is the key idea. Because a force is felt only when the car is turning, we can detect turns on our path by measuring the forces we feel along the path.\\
	\\
	Those familiar with Newton's laws know that the force of an object is given by the object's mass times it's acceleration. The acceleration of an object is the second derivative of the object's position. 
	
\end{document}
