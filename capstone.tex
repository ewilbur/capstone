\documentclass[]{article}
\usepackage{amsmath,amsthm,amssymb,amsfonts}
\usepackage[shortlabels]{enumitem}
\usepackage{xcolor}

\usepackage{geometry}
\geometry{
	left=2cm,
	right=2cm
}

% Package used for creating hyperlinks
%\usepackage{hyperref}
%\hypersetup{
	%	colorlinks=true,
	%	linkcolor=blue,
	%	citecolor=red,
	%	urlcolor=blue,
	%	pdfnewwindow=true
	%}

% Package for graphs, diagrams, and charts
%\usepackage{tikz}
%\usepackage{tikz-cd}
%\graphicspath{ {./images/} }

\renewcommand{\Pr}[1]{\text{Pr}\left(#1\right)} % For probability to look like Pr(...)

\let\oldemptyset\emptyset % Rebind the crappy looking emptyset to another variable
\let\emptyset\varnothing % and redfine emptyset as the good looking one

\let\oldphi\phi % Rebind the crappy looking phi to another variable
\let\phi\varphi % and redefine phi as the good looking one


\newcommand\<{\ensuremath{\left\langle}}
\renewcommand\>{\ensuremath{\right\rangle}}
% Common math environments
\newcommand{\kk}{\ensuremath{\Bbbk}} 
\newcommand{\CC}{\ensuremath{\mathbb{C}}} 
\newcommand{\NN}{\ensuremath{\mathbb{N}}}
\newcommand{\QQ}{\ensuremath{\mathbb{Q}}} 
\newcommand{\RR}{\ensuremath{\mathbb{R}}}
\newcommand{\FF}{\ensuremath{\mathbb{F}}}
\newcommand{\ZZ}{\ensuremath{\mathbb{Z}}} 
\newcommand{\cO}{\ensuremath{\mathcal{O}}} 
\newcommand{\Aut}{\ensuremath{\mathrm{Aut}}} 
\newcommand{\Gal}{\ensuremath{\mathrm{Gal}}} 


% Solution environment. Basically just a standard proof environment but titled as "solution"
\newenvironment{solution}
{
	\begin{proof}[Solution] \text{ }
		\\
	}
	{
	\end{proof}
}

\newenvironment{graybox}{%
	\begin{lrbox}{\grayboxcontent}%
		\begin{minipage}{\textwidth}%
		}{%
		\end{minipage}%
	\end{lrbox}%
	\colorbox{gray!20}{\usebox{\grayboxcontent}}%
}

\newsavebox{\grayboxcontent} % Create a savebox to store the content

%Light gray box is for subproblems
\newenvironment{lightgraybox}{%
	\begin{lrbox}{\lightgrayboxcontent}%
		\begin{minipage}{\dimexpr\linewidth-2\fboxsep-2\fboxrule}%
		}{%
		\end{minipage}%
	\end{lrbox}%
	\colorbox{gray!10}{\usebox{\lightgrayboxcontent}}%
}

\newsavebox{\lightgrayboxcontent} % Create a savebox to store the content


%opening
\title{A Visual Introduction to Curvature}
\author{Evan Curry Wilbur}

\begin{document}
	\maketitle
	
	\begin{abstract}
		This paper seeks to cover what one would learn in a years long graduate course in Riemannian geometry in a way that is both visually stimulating and enlightening. It is directed primarily to early undergraduate students of math who have familiarity with topics from linear algebra. Much reliance will be given to the reader's understanding of inner product spaces, determinants, linear transformations, as well as a familiarity with the geometric understanding of all these ideas. However, a precocious high schooler could follow along after reading chapters 1,2,3, and 6 of Shelden Axler's Linear Algebra Done Right. As well, practice in computing derivatives and integrals and the geometric meaning therein would be helpful. However, a course in vector calculus is not required.
		\\
		\\
		\indent
		By no means should this be seen as a full and rigorous exploration into the vast topic of Riemannian and differential geometry; many tomes have be written doing such a task. Instead, it will leverage visual intuition of the space we occupy to make broad and powerful statements about surfaces more generally. Of course, visual intuition is limited by the dimension we occupy, so many generalizations will be stated, without proof, with references to where one could find proofs.
	\end{abstract}
	
	\section{Manifolds}
	\subsection*{Charts}
	
	\subsection*{Manifold}
	
	\subsection*{Rough and Smooth Manifolds}
	
	\section{The Measure of a Manifold}
	\subsection*{The Euclidean Metric}
	\subsection*{The Riemannian Metric}
	
	\section{Connections}
	\subsection*{Vector Fields}
	\subsection*{Geodesics}
	
	\section{Curvature}
	
	
	
	
\end{document}
